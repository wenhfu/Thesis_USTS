\documentclass[cs4size]{ctexbook}
\usepackage{amsmath,amssymb}
\usepackage{hyperref}  % 生成目录的超链接
\hypersetup{
	colorlinks=true,  % 启用颜色链接
	linkcolor=black,  % 目录链接颜色设为黑色
	urlcolor=black,   % URL 颜色设为黑色
	citecolor=black   % 引用颜色设为黑色
}
\usepackage[
top=2.54cm,      % 上边距
bottom=2.54cm,   % 下边距
left=3.17cm,     % 左边距
right=3.17cm     % 右边距
]{geometry}

\usepackage{lipsum} % 随机产生英文文本
\usepackage{zhlipsum} % 随机产生中文文本

\usepackage{fancyhdr} % 引入 fancyhdr 宏包
\pagestyle{fancy}     % 启用 fancy 页面样式
\renewcommand{\headrulewidth}{0.4pt} % 设置页眉横线粗细
\renewcommand{\headrule}{%
	\hrule width\headwidth height\headrulewidth \vskip-\headrulewidth%
}
\fancyhead[L]{} % 左页眉为空
\fancyhead[C]{苏州科技大学本科生毕业论文} % 居中显示章节标题
\fancyhead[R]{} % 右页眉为空

\usepackage{enumitem} % 全局设置 itemize 行间距
\setlist[itemize]{
	itemsep=0.1\baselineskip, % 项目之间的间距
	topsep=0.1\baselineskip,  % 列表与上下文的间距
	parsep=0.1\baselineskip   % 段落之间的间距
}

\usepackage{setspace}  % 设置行距
\fontsize{12pt}{18pt}\selectfont  % 设置小四字体(12pt)和1.5倍行距(18pt)
\setstretch{1.5}  % 设置1.5倍行距
\linespread{1.5}\selectfont % 设置行距

\CTEXsetup[beforeskip={0.5\baselineskip},afterskip={0.5\baselineskip},format+={\heiti\zihao{-3}}]{chapter}
\CTEXsetup[format+={\raggedright\heiti\zihao{4}}]{section}
\CTEXsetup[format+={\raggedright\heiti\zihao{-4}}]{subsection}

\usepackage{algorithm,algorithmic}
\makeatletter
\newenvironment{breakablealgorithm}
{% \begin{breakablealgorithm}
		\footnotesize
		\vskip 4mm
		\refstepcounter{algorithm}% New algorithm
		\hrule height.8pt depth0pt \kern2pt% \@fs@pre for \@fs@ruled
		\vskip 0.5mm\renewcommand{\caption}[2][\relax]{% Make a new \caption
			{\raggedright\textbf{\ALG@name~\thealgorithm} ##2\par}%
			\ifx\relax##1\relax % #1 is \relax
			\addcontentsline{loa}{algorithm}{\protect\numberline{\thealgorithm}##2}%
			\else % #1 is not \relax
			\addcontentsline{loa}{algorithm}{\protect\numberline{\thealgorithm}##1}%
			\fi
			\vskip 0.5mm\kern2pt\hrule\kern2pt\vskip 1mm
		}
	}{% \end{breakablealgorithm}
	\kern2pt\hrule height.8pt depth0pt\relax\vskip 4mm% \@fs@post for \@fs@ruled
}
\makeatother

\usepackage{titlesec} % 设置标题格式
\titlespacing*{\section}{0pt}{3pt}{3pt}

\numberwithin{equation}{section}


\begin{document}
	\zihao{-4}
	
	
	\title{题目}
	\author{姓名}
	\date{}
	\maketitle
	
	\chapter*{摘要}
	\thispagestyle{fancy} % 强制当前页显示页眉
	\addcontentsline{toc}{chapter}{摘要}
	\pagenumbering{Roman} 
	
	\zhlipsum[1]
	
	\chapter*{Abstract}
	\addcontentsline{toc}{chapter}{Abstract}
	\thispagestyle{fancy} % 强制当前页显示页眉
	
	\lipsum[1]
	
	\tableofcontents  % 生成目录
	\thispagestyle{fancy} % 强制当前页显示页眉
	
	\chapter{引言}
	\thispagestyle{fancy} % 强制当前页显示页眉
	\pagenumbering{arabic} 
	
	主要来自于开题报告。毕业论文正文部分完成后,再补充。
	
	
	\section{研究背景}\indent

	\zhlipsum[1-3]
	
	\section{研究内容}\indent
	
	\zhlipsum[1-3]
	
	\section{研究目的和意义}\indent
	
	\zhlipsum[1-3]
	
	\chapter{章节名1}
	\thispagestyle{fancy} % 强制当前页显示页眉
	
	\section{子标题1}\indent
	
	\zhlipsum[1-3]
	
	\section{子标题2}\indent
	
	\zhlipsum[1-3]
	
	\section{子标题3}\indent
	
	\zhlipsum[1-3]
	
	\section{子标题4}\indent
		
	\zhlipsum[1-3]
	
	\chapter{章节名2}
	\thispagestyle{fancy} % 强制当前页显示页眉
	
	
	\section{子标题1}\indent
	
	\zhlipsum[1-3]
	
	\section{子标题2}\indent
	
	\zhlipsum[1-3]
	
	\section{子标题3}\indent
	
	\zhlipsum[1-3]
	
	\chapter{总结}
	\thispagestyle{fancy} % 强制当前页显示页眉
	
	\zhlipsum[1-3]
	
	\chapter*{参考文献}
	\addcontentsline{toc}{chapter}{参考文献}
	\thispagestyle{fancy} % 强制当前页显示页眉
	
	
\end{document}